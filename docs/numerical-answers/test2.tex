\documentclass{article}
\usepackage[utf8]{inputenc}
\usepackage{amsmath}

\title{Numerical Reasoning Test Solutions Test 2}
\author{Jaklyn Crilly}
\date{}

\begin{document}

\maketitle

$\textbf{Question 1} \\$
The revenue (in units of Euro) from the sale of Widgets for a specified month, denoted $R_{\text{Month}}$, is given by the sum of the revenue produced from the three widgets ABX56, ABX66 and ABX76, of the specified month. That is:
\begin{align*}
R_{\text{April}} &= 600 + 546 + 280\\
&= 1426,\\
R_{\text{May}} &= 690 + 546 + 360\\
&= 1596,\\
R_{\text{June}} &= 780 + 650 + 480\\
&= 1910.
\end{align*}
The total revenue over these three months is the sum of these monthly total revenues, and so:
\begin{align*}
R_{\text{Total}} &= 1426 + 1596 + 1910\\
&= 4932.
\end{align*}
The percentage of total sales revenue generated in May, denoted $P_{\text{May}}$, is then given by the total revenue from the sales of Widgets in May ($R_{\text{May}}$) expressed as a percentage of the total revenue from the sale of Widgets across all three months ($R_{\text{Total}}$):
\begin{align*}
P_{\text{May}} &= \frac{1596}{4932}\times 100\\
&= 32.36...
\end{align*}
To the nearest percent, we get that the answer is 32$\%$. $\\$

$\textbf{Question 2} \\$
For a given month, the price of a single widget, denoted $C_{\text{month}}^{\text{widget}}$, is given by the revenue made by that widget divided by the total number of sales of that widget for the specified month. For the cases of ABX56 and ABX66 in June, we get:
\begin{align*}
C_{\text{June}}^{\text{ABX56}} &= \frac{780}{52}\\
&= 15,\\
C_{\text{June}}^{\text{ABX66}} &= \frac{650}{50}\\
&= 13.
\end{align*}
So in June, the revenue made from 1 ABX56 widget was €15 and the revenue made from 1 ABX66 widget was €13. Therefore, the revenue made from 8 ABX56 widgets is $8 \times €15 = €120$, and from 9 ABX66 widgets is $9 \times €13 = €117$. $\\$

The sale revenue generated by selling 8 units of ABX56 and 9 units of ABX66 in June is thus $€120+€117=€237$. $\\$

$\textbf{Question 3} \\$
The cost of revenue for a single ABX76 widget (i.e. its unit price) in June was $\frac{480}{24}=20$ Euros. If in July this unit price decreased by 20$\%$, this means that the unit price of an ABX76 widget in July, denoted $C_{\text{ABX76}}^{\text{July}}$, is:
\begin{align*}
C_{\text{ABX76}}^{\text{July}} &= 20 - 20 \times \frac{20}{100}\\
&= 16.
\end{align*}
The number of units of ABX76 sold in June is 24. If the sales increased by 25$\%$ from June to July, this means that the total number of units of ABX76 sold in July, denoted $N_{\text{ABX76}}^{\text{July}}$, is:
\begin{align*}
N_{\text{ABX76}}^{\text{July}} &= 24 \times \frac{100 + 25}{100}\\
&= 30.
\end{align*}
The total July sales revenue generated by the ABX76 widget (in Euros), denoted $R_{\text{ABX76}}^{\text{July}}$, is given by the total number of units sold multiplied by the cost of an individual sale, and so we get:
\begin{align*}
R_{\text{ABX76}}^{\text{July}} &= 30 \times 16\\
&= 480.
\end{align*}
The answer is €480. $\\$

$\textbf{Question 4} \\$
In 1985 the number of fares collected for subways was 350 million and in 1987 it was 260 million. The percentage increase in fare collections over this period, denoted $P_I$, is given by the increase in the number of fares collected from 1987 to 1985, expressed as a fraction over the initial number of fares collected (i.e. fares collected in 1985). That is:
\begin{align*}
P_I &= \frac{260-350}{350} \times 100\\
&= -25.71...
\end{align*}
So we see that there was a percentage increase of -25$\%$ from 1985 to 1987. The negative sign indicated a percentage decrease, and so to the nearest percent, we get that the percentage decrease in fares collected from 1985 to 1987 was 25$\%$. $\\$

$\textbf{Question 5} \\$
In 1984, there were 350 million subway fares collected, and 310 million bus fares collected. $\\$

Given the average subway fare collected was 50 cents = $\$$0.5, the total dollar amount of subway fares collected was:
$$\$0.5 \times 350 \times 10^6 = \$175 \times 10^6.$$

Similarly, given the average bus fare collected was 30 cents = $\$$0.3, the total dollar amount of bus fares collected was:
$$\$0.3 \times 310 \times 10^6 = \$93 \times 10^6.$$

The required ratio is therefore:
\begin{align*}
175 \times 10^6 : 93 \times 10^6 &= 175:93\\
&= 1 : \frac{93}{175}\\
&= 1 : 0.5314...
\end{align*}

Now, to compare this solution to the possible answers we will multiply both sides of the ratio by the value of the first position of each ratio in the solutions. Doing so, we get:
\begin{align*}
1: 0.53 &= 11 : 5.8457...\\
&= 17 :9.0342...\\
&= 5 : 2.6571...\\
&= 19 : 10.0971...
\end{align*}
Comparing these ratios to the ratios of the solutions, we see that to the nearest whole number the ratio is approximately $19 : 10$, and thus, this is the solution. $\\$

$\textbf{Question 6} \\$
In 1987, 440 million commuter rail fares were collected. For this year, the total number of fares collected on subways, commuter rails and buses was $260+440+245=945$ million. The percentage of fares collected on commuter rails, denoted $P_{CR}$, is thus:
\begin{align*}
P_{CR} &= \frac{440}{945} \times 100\\
&= 46.5608...
\end{align*}
Therefore, the answer is approximately 47$\%$. $\\$

$\textbf{Question 7} \\$
The domestic sales in year Y totaled £10 million, and in year Y+3 totaled £16 million. This means that the domestic sales increased by £6 million over this period.$\\$

The percentage increase in domestic sales, denoted $P_{\text{DS}}$, is then given by expressing the increase in domestic sales as a percentage over the total domestic sales in year Y. That is:
\begin{align*}
P_{\text{DS}} &= \frac{6}{10} \times 100\\
&= 60.
\end{align*}
The answer is 60$\%$. $\\$

$\textbf{Question 8} \\$
The average yearly operating expenses of Corporation X (in units of millions of £), denoted $E_{OE}$, is given by the sum of the yearly operating expenses from year Y up to and including year Y+4, divided by the number of years over this period (that is, divided by 5). So we get:
\begin{align*}
E_{OE} &= \frac{8 + 14 + 18 + 19 + 21}{5}\\
&= 16.
\end{align*}
Therefore, the average yearly operating expenses of Corporation X was £16 million = £16,000,000. $\\$

$\textbf{Question 9} \\$
In year Y+3 the total domestic sales revenue was £16 million, whilst the total foreign sales revenue was £10 million. The total revenue was thus $£16 + £10 = £26$ million. The fraction of the revenue that was due to domestic sales, denoted $F_D$, and foreign sales, denoted $F_F$, is thus:
\begin{align*}
F_D &= \frac{16}{26},\\
F_F &= \frac{10}{26}.\\
\end{align*}
Now, given that a pie chart has a total of 360 degrees, the angle of the pie chart representing domestic sales, denoted $D_D$, and the angle representing foreign sales, $D_F$, is:
\begin{align*}
D_D &= 360 \times F_D\\
&= 221.53...\\
D_F &= 360 \times F_F\\
&= 138.46...
\end{align*}
The pie chart that is closest to representing 222 degrees worth of domestic sales and 138 degrees worth of foreign sales is graph D). $\\$

$\textbf{Question 10} \\$
$\textbf{Comment: }$ Keep solution exactly as is. $\textbf{Comment end.}$ $\\$

$\textbf{Question 11} \\$
For regional office C, there is a daily average of 180 long distance domestic calls. Given the average cost of each of these calls is €4.60, the average daily cost (in €) of long distance domestic calls in regional office C, denoted $C_C^{\text{LDD}}$, is:
\begin{align*}
C_C^{\text{LDD}} &= €4.60 \times 180\\
&= €828.
\end{align*}
So the cost of these daily calls is €828, which is approximately €830. $\\$

$\textbf{Question 12} \\$
$\textbf{Comment: }$ Keep solution exactly as is, however include the following as the first paragraph. $\textbf{Comment end.}$ $\\$

For any given region, the percentage of calls which were local is given by the number of local calls divided by the sum of the number of local calls from each office, then multiplied by 100 to get the expression in percentage form. $\\$

$\textbf{Question 13} \\$
$\textbf{Comment: }$ Keep solution as is, however delete the last paragraph which starts 'The fraction...' and replace it with the following. $\textbf{Comment end.}$ $\\$

The fraction of average daily expenditure due to international long distance calls for office B, denoted $F_B^{\text{ILD}}$, is given by the cost of international long distance calls from office B, divided by the sum of the cost of all types of calls for office B. That is:
\begin{align*}
F_B^{\text{ILD}} &= \frac{3924}{6975}\\
&= 0.5625...
\end{align*}
Now expressing the solutions in decimal form:
\begin{align*}
\frac{2}{5} &= 0.4,\\
\frac{11}{20} &= 0.55,\\
\frac{2}{3} &= 0.33...,\\
\frac{3}{4} &= 0.75,\\
\frac{17}{20} &= 0.85,
\end{align*}
we see that the solution is 0.5625..., which is approximately $\frac{11}{20}$. $\\$

$\textbf{Question 14} \\$
Using the same table as created for the previous question, the ratio of average daily cost of calls at office C to average daily costs of calls at office E is:
\begin{align*}
    2976 : 3002.7 &= 1 : \frac{3002.7}{2976}\\
    &= 1 : 1.0089...
\end{align*}
So we see that this ratio is $\frac{1.0089..}{1}=1.0089 \approx 1$. $\\$

$\textbf{Question 15} \\$
The gross receipts for year 1 were €7.5 million, whilst the sum of the gross receipts for years 1, 2 and 3 (in millions of €) were:
$$7.50 + 8.55 + 8.10 = 24.15.$$
The gross receipts for year 1 expressed as a percentage over the total gross receipts for all three years is thus:
\begin{align*}
\frac{7.5}{24.15} \times 100 &= 31.0559...
\end{align*}
So to the nearest percent, we get that the answer is 31$\%$. $\\$

$\textbf{Question 16} \\$
The gross receipts for year 3 was €8.1 million. Given the category "Other" represents the countries C4 and C5, and in year 3 the "Other" countries accounted for 10$\%$ of the gross receipts, we get that the total value of gross receipts for C4 and C5 is:
$$€8100000 \times \frac{10}{100} = €810000.$$
Therefore, the answer is €810,000. $\\$

$\textbf{Question 17} \\$
$\textbf{Comment: }$ Keep solution as is, however include the following as a first paragraph. $\textbf{Comment end.}$ $\\$

If a country has $x\%$ of the total gross receipts (TGR), the total gross receipts for this country, denoted $\text{TGR}_{\text{C}}$, is:
$$\text{TGR}_{\text{C}} = \text{TGR} \times \frac{x}{100}.$$

$\textbf{Question 18} \\$
By looking at the second graph with vertical label `Total Profits', we see that the only year in which profits from food-related operations increased from its previous year was for the year Y+4. Now looking at the first graph, we see that the total revenue for year Y+4 was £7.9 billion.
Therefore, the answer is approximately £8.0 billion. $\\$

$\textbf{Question 19} \\$
In the year Y+4, GiantMart's total revenue was £7.9 billion, and the total profits were £800 million = £0.8 billion. The percentage of the total revenues that was due to the total profits, which we denote $P$, is thus:
\begin{align*}
P &= \frac{0.8}{7.9} \times 100\\
&= 10.1265...
\end{align*}
So we see that the answer is approximately 10$\%$.$\\$

$\textbf{Question 20} \\$
Focusing on the first graph, we see that the revenue (in units of billions of £) from non-food related operations (indicated by the colour blue) is:
\begin{align*}
\text{Year     Y  } &: 2.9-2.8 =0.1\\
\text{Year Y+1} &: 4.9-2.9=2.0\\
\text{Year Y+2} &: 6.0-3.5=2.5\\
\text{Year Y+3} &: 7.5-2.8=4.7.
\end{align*}
So we see that the first year in which this revenue exceeds £4.5 billion is Y+3. $\\$

Now focusing on the second graph, we see that the total profits in the year Y+3 was £800 million. $\\$

$\textbf{Question 21} \\$
In the year Y+5, the total revenue from food-related operations was £2.9 billion. Of this revenue, a total of 20$\%$ was a result of revenue from frozen food operations. This means that in units of billions of £, the total revenue from frozen foods was:
$$2.9 \times \frac{20}{100} = 0.58.$$
Therefore, a total of £0.58 billion = £580 million of the revenue from the year Y+5 was from frozen food operations, and so the answer is 580.

\end{document}

