\documentclass{article}
\usepackage[utf8]{inputenc}
\usepackage{amsmath}

\title{Numerical Reasoning Test Solutions Test 10}
\author{Jaklyn Crilly}
\date{}

\begin{document}

\maketitle

$\textbf{Question 1} \\$
In 2006, the GDP for Luxembourg was $\$$41 million, whilst the GDP in the USA was $\$$13,202 million. How many times the USA's GDP was bigger than Luxembourg's GDP, is given by the USA's GDP expressed as a fraction over Luxembourg's GDP, so we get that this is:
\begin{align*}
\frac{13202}{41} = 322,
\end{align*}
and so the answer is 322 times bigger. $\\$

$\textbf{Question 2} \\$
To calculate the GDP per capita, which we denote GDP/P, we must divide the GDP by the population. Doing this for each of the six countries, we get:
\begin{align*}
(\text{GDP/P})_{\text{USA}} &= \frac{13202}{298.99} = 44.1...,\\
(\text{GDP/P})_{\text{Japan}} &= \frac{4340}{127.56}= 34.0...,\\
(\text{GDP/P})_{\text{UK}} &= \frac{2345}{60.36}= 38.8...,\\
(\text{GDP/P})_{\text{Switz}} &= \frac{380}{7.45}= 51.0...,\\
(\text{GDP/P})_{\text{Norway}} &= \frac{311}{4.64}= 67.0...,\\
(\text{GDP/P})_{\text{Lux}} &= \frac{41}{0.46}= 89.1...
\end{align*}
Therefore,  Luxembourg had the highest GDP per capita in 2006, with a value of $\$$89.1 dollars per head. $\\$

$\textbf{Question 3} \\$
$\textbf{Comment: }$ I think the answer is wrong. Shouldn't be in millions, as the millions cancel. $\textbf{Comment end.}$ $\\$

The mean GDP per capita for all six countries surveyed, denoted GDP/P, is given by the total GDP of the six countries, divided by the sum of the total population of the six countries.$\\$

The total population (in millions), denoted P, is:
\begin{align*}
\text{P} &= 298.99+127.56+60.36+7.45+4.64+0.46\\
&=499.46.
\end{align*}

The total GDP (in millions) is:
\begin{align*}
\text{GDP} &= 13202+4340+2345+380+311+41\\
&= 20619.
\end{align*}

Therefore, the total GDP per capita for all six countries surveyed is:
\begin{align*}
\text{GDP/P} &= \frac{20619}{499.46}\\
&= 41.2825...
\end{align*}
To two decimal places, the answer is $\$$41.28 per head. $\\$ 

$\textbf{Question 4} \\$
$\textbf{Comment: }$ Keep solution as is. However, include the following as a first paragraph. $\textbf{Comment end.}$ $\\$

The range of a given commodity over the period of August 2009 to July 2010 is equal to the maximum index value for the commodity over this period, minus the minimum index value of the commodity over this period. $\\$


$\textbf{Question 5} \\$
$\textbf{Comment: }$ Keep solution as is. However, include the following as a first paragraph. $\textbf{Comment end.}$ $\\$

Letting the value of the index for a given commodity be denoted $I_A$ and $I_F$ for the months of August 2009 and February 2010 respectively, the percentage increase in the value of the commodity's index over this period, denoted $P_I$, is given by the formula:
\begin{align*}
P_I &= \frac{I_F - I_A}{I_A} \times 100,
\end{align*}
where $I_F - I_A$ is the increase in the index between August 2009 and February 2010. $\\$

$\textbf{Question 6} \\$
The percentage increase in vehicle production from 2004 to 2009, denoted $P_I$, is given by the change in vehicle production over this period (i.e. vehicle production in 2009 (denoted $N_{2009}$) minus the vehicle production in 2000 ($N_{2000}$)), expressed as a percentage of the vehicle production in 2000. That is:
$$P_I = \frac{N_{2009} - N_{2000}}{N_{2000}} \times 100.$$

Now analysing the graph, we see that Japan's production decreased from 2000 to 2009, and so this cannot be the answer. For the remaining four countries that are given as potential solutions, we get:
\begin{align*}
P_I^{\text{India}} &= \frac{2.6-1.5}{1.5} \times 100\\
&= 73.3333...,\\
P_I^{\text{Thailand}} &= \frac{1.0-0.9}{0.9} \times 100\\
&= 11.1111...,\\
P_I^{\text{Turkey}} &= \frac{0.9-0.8}{0.8} \times 100\\
&= 12.5,\\
P_I^{\text{Iran}} &= \frac{1.4-0.8}{0.8} \times 100\\
&= 75.
\end{align*}

So apart from China, the greatest percentage increase in vehicle production occurred in Iran with a percentage increase of 75$\%$. $\\$

$\textbf{Question 7} \\$
First we need to work out China's production for the year 2004, denoted $P_{2004}^{\text{China}}$, and 2009, denoted $P_{2004}^{\text{China}}$, as percentages of the total vehicle production for the seven countries of the given year, denoted $N_{\text{year}}$. $\\$

Calculating the vehicle production numbers, we get:
\begin{align*}
N_{2004} &= 10.5+5.1+3.5+1.5+0.9+0.8+0.8\\
&= 23.1,\\
N_{2009} &= 7.9+13.8+3.5+2.6+1.+0.9+1.4\\
&= 31.1.
\end{align*}
Now we can calculate the percentages for the two years:
\begin{align*}
P_{2004}^{\text{China}}&=\frac{5.1}{23.1} \times 100\\
&= 22.0779...\\
P_{2009}^{\text{China}}&= \frac{13.8}{31.1} \times 100\\
&= 44.3729...
\end{align*}
So we see that the percentage of vehicles produced in China in the year 2004 is 22$\%$, and in the the year 2009 is 44$\%$. Therefore, China's portion of vehicle production increased by a factor of $\frac{P_{2009}^{\text{China}}}{P_{2004}^{\text{China}}}= 2.0098...$, and so the answer is approximately 2 times. $\\$

$\textbf{Question 8} \\$
A decrease of 5.6$\%$ each year means that for each year after 2009, Japan's vehicle production will be $100\% -5.6\% = 94.4\%$ of its previous year's production. Given Japan's total production (in units of million) was 7.9 for the  year 2009, Japan's production (in units of million) for the year 2010, denoted $P_{2010}$, is:
\begin{align*}
P_{2010} &= 7.9 \times \frac{94.4}{100}\\
&= 7.4576.
\end{align*}
Japan's production (in units of million) for the year 2011, denoted $P_{2011}$, is then:
\begin{align*}
P_{2010} &= 7.4576 \times \frac{94.4}{100}\\
&= 7.0399744.
\end{align*}
So, to the nearest 0.1 million, we get that Japan's vehicle production in the year 2011 is 7.0 million. $\\$

$\textbf{Question 9} \\$
South Korea's share of the world production of vehicles for a given year, denoted $P_{\text{SK}}^{\text{year}}$, is given by South Korea's vehicle production for the given year expressed as a percentage over the total world production of vehicles for that year. Calculating this quantity, we get:
\begin{align*}
P_{\text{SK}}^{2004} &= \frac{3.5}{64.2} \times 100\\
&=5.4517... \\
P_{\text{SK}}^{2004} &= \frac{3.5}{61.7} \times 100\\
&= 5.6726...
\end{align*}
We see that South Korea's share of the world production of vehicles increased by $5.6726...\%-5.4517...\%=0.2208...\%$, and so the answer is about 0.2$\%$. $\\$

$\textbf{Question 10} \\$
$\textbf{Comment: }$ The first line says 'The above table' but the table is below. In the fourth paragraph there should be a comma after cigarettes. $\textbf{Comment end.}$ $\\$

The number of male smokers in the year 2007 for a given age range (in units of million), which we denote by $N_S^{\text{age}}$, is given by the total male population of this age range (as given by the responses weighted by population), multiplied by the fraction of smokers of any kind (which is given by the percentage of smokers of any kind divided by 100). Doing this for the 5 age groups given as potential solutions, we get:
\begin{align*}
N_S^{20-24} &= 1255 \times \frac{33}{100}\\
&= 414.15,\\
N_S^{25-29} &= 1367 \times \frac{34}{100}\\
&= 464.78,\\
N_S^{30-34} &= 1653 \times \frac{27}{100}\\
&= 446.31,\\
N_S^{35-49} &= 5738 \times \frac{26}{100}\\
&= 1491.88,\\
N_S^{\text{over}60} &= 5424 \frac{15}{100}\\
&= 813.6.
\end{align*}
The highest number of male smokers occurred in the age range 35-49 with a total of 1,491.88 million smokers. $\\$

$\textbf{Question 11} \\$
$\textbf{Comment: }$ I think the solution is wrong. The responses weighted by population are in millions, not thousands. $\textbf{Comment end.}$ $\\$

As given in the graph, we know that the population of men aged 16 and over totals 19996 million.
Of this 19996 million, 1$\%$ smoked pipes in 2007. This means that a total of:
\begin{align*}
19996 \times 10^6 \times \frac{1}{100} = 199960000
\end{align*}
men aged 16 or above smoked pipes in 2007. So to the nearest 10000, we get that the answer is 199,960,000. $\\$

$\textbf{Note: }$ If you work out the solutions by summing the separate figures for the age groups 16-19, 50-59, and 60 and over, the answer obtained is 187 million. This however, gives a less accurate answer as a result of the percentages being rounded to the nearest whole number. $\\$

$\textbf{Question 12} \\$
$\textbf{Comment: }$ I think the solution is wrong. The responses weighted by population are in millions, not thousands. $\textbf{Comment end.}$ $\\$

The number of men in the age group 30-34 that smoked cigarettes is 26$\%$ of the total 1653 million (30-34 year old) men, which is equal to $1653 \times \frac{26}{100}=429.78$ million men. $\\$

The number of men in the age group 20-24 that smoked cigarettes is 32$\%$ of the total 1255 million (20-24 year old) men, which is equal to $1255 \times \frac{32}{100} = 401.60$ million men. $\\$

The number of more men that smoke in the age group of 30-34 year old men than in the age group 20-34 is
$429.78-401.60=28.18$ million. $\\$

$\textbf{Question 13} \\$
There were 2400 men in the age group of 60 and over that took the survey. We know that 15$\%$ of these men smoke, which means that of the men in the age group of 60 and over that took the survey, $2400 \times \frac{15}{100} = 360$ of them were smokers. $\\$

$\textbf{Question 14} \\$
In 1993, the number of people in rural South Asia who lived below the $\$$1-a-day poverty line was roughly 110 million, whilst the number of people living in urban South Asia who lived below the $\$$1-a-day poverty line was roughly 390 million. $\\$

Therefore, the total number of people in South Asia who lived below the $\$$1-a-day poverty line was approximately $110+390=500$ million people. $\\$

$\textbf{Question 15} \\$
The number of people in rural areas of East Asia and the Pacific who lived below the $\$$1-a-day poverty line in 1993 was roughly 400 million, and in 2002 was roughly 220 million. So we see that this number of people dropped by roughly $400-220=180$ million people between 1993 and 2002.$\\$

The percentage change, denoted $P_C$, is then given by the fall in population, expressed as a percentage over the initial number of people (that is, the number of people in rural areas of East Asia and the Pacific who lived below the $\$$1-a-day poverty line in 1993):
\begin{align*}
P_C &= \frac{180}{400} \times 100\\
&= 45.
\end{align*}
Therefore, the answer is a fall of roughly 45$\%$. $\\$

$\textbf{Question 16} \\$
$\textbf{Comment: }$ Keep everything in the solution up to and including the table. Delete everything after it and replace it with the following. $\textbf{Comment end.}$ $\\$

From the table we see that there was a total of 300 million people living below the $\$$1-a-day poverty line in 2002. The fraction of these people that lived in a given region, denoted $F_{\text{region}}$, is equal to the number of people living below the $\$$1-a-day poverty line in this region, divided by the total number of such people over all regions (which we know is roughly 300 million).  $\\$

Now, given a pie chart has a total of 360 degrees, if the urban data was represented in a pie chart, the number of degrees (i.e. the total angle in units of degrees) a region would occupy, denoted $A_{\text{region}}$, is given by the fraction of the people living in this region ($F_{\text{region}}$) multiplied by 360. That is:
$$A_{\text{region}} = F_{\text{region}} \times 360.$$

For the six given regions, we get:
\begin{align*}
A_{\text{SSA}} &= \frac{100}{300} \times 360\\
&= 120,\\
A_{\text{SA}} &= \frac{120}{300} \times 360\\
&= 144,\\
A_{\text{EAP}} &= \frac{20}{300} \times 360\\
&= 24,\\
A_{\text{MENA}} &= \frac{10}{300} \times 360\\
&= 12,\\
A_{\text{ECA}} &= \frac{10}{300} \times 360\\
&= 12,\\
A_{\text{LAC}} &= \frac{40}{300} \times 360\\
&= 48.
\end{align*}
The region of Latin America and Caribbean (LAC) produces an angle of roughly 48 degrees, which is clearly the closest angle to 50 degrees compared to all the other regions and is thus, the answer. $\\$

$\textbf{Question 17} \\$
We know that in 2000, the total amount of greenhouse gas emissions globally was estimated to be 42 GtCO2e. A total of 21.3$\%$ of these greenhouse gases were emitted by power stations, so letting $E_{PS}$ denote the amount of greenhouse gas emissions arising due to power stations in 2000 (in units of GtCO2e), we get:
\begin{align*}
E_{PS} &= 42 \times \frac{21.3}{100}\\
&= 8.946.
\end{align*}
So 8.946 GtCO2e, which is equivalent to 8.946 billion metric tonnes of CO2 equivalent, of greenhouse gases were emitted by power stations. Rounding this to the nearest billion, we get that the answer is 9 billion. $\\$

$\textbf{Question 18} \\$
We know that in the year 2000, the total amount of greenhouse gas emissions globally was estimated to be 42 GtCO2e. Now of the 42 GtCO2e of greenhouse gases emitted, 9$\%$ was Nitrous Oxide. So the total amount of Nitrous Oxide emitted (in units of GtCO2e), denoted $E_{N2O}$, is:
\begin{align*}
E_{N2O} &= 42 \times \frac{9}{100}\\
&= 3.78.
\end{align*}
Of this 3.78 GtCO2e of Nitrous Oxide emitted, 62$\%$ of it was due to agricultural byproducts. This means that the amount of Nitrous Oxide emitted due to agricultural byproducts (in units of GtCO2e), denoted $E_{N2O}^{AB}$, is:
\begin{align*}
E_{N2O}^{AB} &= 3.78 \times \frac{62}{100}\\
&= 2.3436.
\end{align*}
So we get 2.3436 GtCO2e, which is equivalent to 2.3436 billion metric tonnes of CO2 equivalent. Rounding this to the nearest billion, we get that the answer is 2 billion. $\\$

$\textbf{Question 19} \\$
Applying the same calculation we did for question 18, however now for the case of Methane and each of the 5 sectors which are given as possible answers, we get that the amount of Methane emitted due to the specified sectors (in units of GtCO2e), denoted $E^{\text{sector}}_{\text{Methane}}$, is:
\begin{align*}
E_{\text{Methane}}^{AB} &= 42 \times \frac{18}{100} \times \frac{40}{100}\\
&=3.024, \\
E_{\text{Methane}}^{FRPD} &= 42 \times \frac{18}{100} \times \frac{29.6}{100}\\
&= 2.23776,\\
E_{\text{Methane}}^{WDT} &= 42 \times \frac{18}{100} \times \frac{18.1}{100}\\
&= 1.36836,\\
E_{\text{Methane}}^{LUBB} &= 42 \times \frac{18}{100} \times \frac{6.6}{100}\\
&= 0.49896,\\
E_{\text{Methane}}^{RCOS} &= 42 \times \frac{18}{100} \times \frac{4.8}{100}\\
&= 0.36288.\\
\end{align*}
We see that the sector `waste disposal and treatment' was responsible for emitting 1.36863 GtCO2e, which is approximately 1.4 GtCO2e of Methane. $\\$

$\textbf{Question 20} \\$
The United States were responsible for 16$\%$ of the global greenhouse gas emissions. Given the total amount of greenhouse gas emissions globally was estimated to be 42 GtCO2e, the total emission of greenhouse gases from the United States (in units of GtCO2e), denoted $E^{\text{US}}$, is:
\begin{align*}
E^{\text{US}} &= 42 \times \frac{16}{100}\\
&= 6.72.
\end{align*}
Now of these emissions, 72$\%$ of them were Carbon Dioxide gases, and so the total amount of Carbon Dioxide that was produced by the United States (in units of GtCO2e), denoted $E^{\text{US}}_{C2O}$, is
\begin{align*}
E^{\text{US}}_{C2O} &= 6.72 \times \frac{72}{100}\\
&= 4.8384.
\end{align*}
So, the amount of Carbon Dioxide produced by the United States in the year 2000 was 4.8384 GtC20e, which to one significant figure, gives an answer of approximately 5 GtC2Oe.
\end{document}

