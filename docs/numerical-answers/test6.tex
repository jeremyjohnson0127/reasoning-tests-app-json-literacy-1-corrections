\documentclass{article}
\usepackage[utf8]{inputenc}
\usepackage{amsmath}

\title{Numerical Reasoning Test Solutions Test 6}
\author{Jaklyn Crilly}
\date{}

\begin{document}

\maketitle

$\textbf{Question 1} \\$
From the graph titled "British Pounds per US Dollar", the vertical axis gives pounds whilst the horizontal gives US dollars, and a point on the graph indicates the equivalent values for GBP and US dollars. Locating $\$$70 on the horizontal axis, find the point on the graph with this horizontal axis value. The point on the vertical axis corresponding to this point on the graph is then £45.5, and so $\$$70 is equivalent to £45.5. $\\$

$\textbf{Question 2} \\$
Focusing on the second graph, find the point £52 along the vertical axis. Reading across, find the point on the graph with this vertical axis coordinate, and then read down to the horizontal axis to determine what US dollar this corresponds to. The graph has a point at £52 whose horizontal coordinate is given by $\$$80.
So £52 is equivalent to $\$$80. $\\$

$\textbf{Question 3} \\$
To answer this question we first need to convert £195 to US dollars, and then convert this US dollars quantity into Euros. $\\$
Focusing on the second graph, we know that it is linear and passes through the two points the origin and the point corresponding to an equivalence of $\$$20 and £13. Letting $y$ denote the vertical axis value (£) of the graph and $x$ the horizontal axis value ($\$$), the equation for this graph is:
$$x=\frac{20}{13} \times y.$$
(Note: this equation for the graph does not depend on our choice of point on the graph corresponding to the equivalence of $\$$20 and £13. Any other point could have been chosen and the same equation would result). $\\$

Letting $y=195$ we find:
\begin{align*}
x &= \frac{20}{13} \times 195\\
&=300.
\end{align*}
So taking $y=£195$, we get $x=\$300$, which is another way of saying that £195 is equivalent to  $\$$300.$\\$

Now we need to work out what $\$$300 is in euros. To do this we apply to the first graph (which is also linear and passing through the origin) the exact same process as we just did to the second graph.$\\$
Choosing the point on the graph corresponding to €50 being equivalent to $\$$68, and again letting $y$ denote the vertical axis value ($\$$) and $x$ the horizontal axis value (€), the equation for the graph is:
$$x=\frac{68}{50} \times y.$$
Taking $y=300$, we find:
\begin{align*}
x &= \frac{50}{68}\times 300\\
&=220.588...
\end{align*}
So we get that £195 is equivalent to $\$$300, which in turn is equivalent to €220.59. The answer is therefore €220.59. $\\$ 

$\textbf{Question 4} \\$
From the second chart we see that $\$$70 is originally worth £45.5, and so we have $\text{£}45.5 = \$70$. $\\$

An increase of 5$\%$ to the GBP against the US dollar means that for each pound, the equivalent value in US dollars is increased by 0.05 times more. Therefore, after this increase, we now have:
$$\text{£}45.5 = \$70 \times 1.05.$$
Dividing both sides of the equation by 1.05, we get:
\begin{align*}
\$70 &= \frac{\text{£}45.5}{1.05}\\
&= \text{£} 43.3333...
\end{align*}
So we see that after the 5$\%$ rise, $\$70$ is equivalent to £$43.33$.
$\\$

$\textbf{Question 5} \\$
The percentage of the $\$$3000 million investment that went to urban and rural projects is given by the sum of their individual percentages, and is thus $13.3\%+12.3\%=25.6\%$. Given $\$$3000 million was invested in total, 25.6$\%$ of this is
$$3000 \times \frac{25.6}{100} = 768.$$
So the total amount of money invested in urban and rural projects is $\$$768 million. $\\$

$\textbf{Question 6} \\$
46.7$\%$ of the total investment was invested into infrastructure, whilst 8.3$\%$ was invested into social welfare. Letting $I_I$ and $I_{SW}$ denote the amount of money invested into infrastructure and social work respectively, given the total amount invested was $\$$3000 million, we find:
\begin{align*}
I_I &= 3000 \times \frac{46.7}{100}\\
&= 1401, \\
I_{SW} &= 3000 \times \frac{8.3}{100}\\
&= 249.
\end{align*}
So $\$$1401 million was invested into infrastructure and $\$$249 million was invested into social welfare.$\\$

The difference in the amount invested in infrastructure compared to social work is:
\begin{align*}
\text{Difference} &= 1401-249\\
&= 1152.
\end{align*}
So $\$$1152 million more is invested in infrastructure than social work. $\\$

$\textbf{Question 7} \\$
The initial amount invested in sustainability is 7$\%$ of the total 3000 million invested, which equates to $3000 \times 0.07 = 210$ million dollars. If this amount is then increased by 5$\%$ we get that the new amount of money invested into sustainability is $210 \times 1.05 = 220.5$ million dollars, which is an increase of $\$$10.5 million.$\\$

Given the investment in all other areas remains the same, the grand total of investments must also increase by $\$$10.5 million, resulting in a new total of $\$$3010.5 million being invested overall.$\\$

The new percentage of investment going to sustainability, denoted $P_S$, is thus:
\begin{align*}
P_S &= \frac{220.5}{3010.5} \times 100\\
&= 7.32...
\end{align*}
Therefore, after the 5$\%$ increase, sustainability now accommodates for 7.32$\%$ of the investment. $\\$

$\textbf{Question 8} \\$
Looking at the Labour row in the chart and comparing the monthly number of intended votes, it is clear that the month with the highest number of intended votes is in March with a total of 314. Given March is not a listed option, the answer is none of these. $\\$

$\textbf{Question 9} \\$
The total number of votes in February, denoted $V^F$, is:
\begin{align*}
V^F &= 388+298+202+132\\
&= 1020.
\end{align*}
The total number of votes that were shared between Labour and Conservative parties, denoted $V_{L+C}$, is:
\begin{align*}
V_{L+C}^F &= 388 + 298\\
&= 686.
\end{align*}
The percentage of votes shared by Labour and Conservative parties in February, $P$, is thus:
\begin{align*}
P &= \frac{V_{L+C}^F}{V^F} \times 100\\
&= \frac{686}{1020}\times 100\\
&= 67.2549...
\end{align*}
So the answer to the nearest percent is 67$\%$. $\\$

$\textbf{Question 10} \\$
The number of votes for the Lib Dem's in December was 210, and in January was 208. There was therefore a decrease of $210-208=2$ votes between December and January. The percentage decrease for this period, labelled $P_C$, is given by the decrease in votes from December to January, expressed as a percentage over the starting number of votes (which is given by the number of Lib Dem votes in December). That is:
\begin{align*}
P_C &= \frac{2}{210} \times 100\\
&= 0.9523...
\end{align*}
So the percentage decrease to two decimal places is 0.95$\%$. $\\$

$\textbf{Question 11} \\$
If the number of votes for the Conservative party is reduced by 8 by the next May poll, given their number of votes in April was 398, there will be 390 votes for the conservative party in May. $\\$

If we want Labour's votes to be equal to this in May, given in April Labour had 312 votes, we require Labour to gain $390-312=78$ votes between April and May. The percentage increase of Labour votes, denoted $P_L$, is then given by:
\begin{align*}
P_L &= \frac{78}{312} \times 100\\
&=25.
\end{align*}
A percentage increase of 25$\%$ in Labour's votes is needed in order for Labour to have an equal number of votes with the Conservative party. $\\$

$\textbf{Question 12} \\$
To determine how much money would be spent on a typical basket of items per week in December 2006, we must calculate the inflation index in December 2006 as a fraction of the inflation index in May 2005, and then multiply this by the initial £400 spent weekly in May 2005 on a typical basket of items. That
is, the amount spent per week on a typical basket of items in December 2006, which we will denote by $S_T$, is:
\begin{align*}
S_T &= \frac{104}{100} \times 400\\
&= 416.
\end{align*}
The answer is £416. $\\$

$\textbf{Question 13} \\$
We want to determine how much a group of items cost in May 2005, which we will denote by the symbol $C_M^{2005}$, given we know that they cost £100 in February 2006. Using the same formula as we applied in question 12, we know that:
\begin{align*}
100 = \frac{100.9}{100} \times C_M^{2005}.
\end{align*}
Rearranging this so that $C_M^{2005}$ is the subject, we find:
\begin{align*}
C_M^{2005} &= \frac{100}{100.9} \times 100\\
&= 99.1080...
\end{align*}
Therefore, to two decimal places, the cost of the group of items was £99.11 in May 2005. $\\$

$\textbf{Question 14} \\$
To determine the percentage increase of a typical basket of items, denoted $P_I$, we need to express the difference in the retail price indices between April 2007 and April 2006 as a percentage of the initial Retail price index over this period (which is the retail price index in April 2006). So we get:
\begin{align*}
P_I &= \frac{104.5 - 101.7}{101.7} \times 100\\
&= 2.753...
\end{align*}
The percentage increase is thus $2.75\%$. $\\$

$\textbf{Question 15} \\$
If we want to make the index at May 2006 the new base, we need to divide all of the indices by the index of May 2016, and then multiply by 100 to get the values back into percentage form. That is, we need to express all values as a percentage of the May 2006 index (note that this results in May 2006 having a new index of 100 as desired). The new index for May 2005, denoted $I_M^{2005}$, is then:
\begin{align*}
I_M^{2005} &= 100 \times \frac{100}{102.3}\\
&=97.7517...
\end{align*}
The answer to one decimal place is thus 97.8$\%$. $\\$

$\textbf{Question 16} \\$
Given we have both the total GDP from Grossland, and how much GDP an individual head of population makes, we can establish the total population by dividing the GDP by the GDP per head of population. The population of Grossland in 2008, denoted $P_{2008}$, is then:
\begin{align*}
P_{2008} &= \frac{1.92 \times 10^{12}}{30089}\\
&= 63,10,694.938...
\end{align*}
where we made sure that both the denominator and the numerator had the same units of currency (both in $\$$) before starting the calculation. $\\$

Therefore, Greenland had a population (to three significant figures) of 63.1 million in 2008. $\\$

$\textbf{Question 17} \\$
The number of the labour force in unemployment, denoted $N_U$, is given by the size of the labour force, denoted $L$, multiplied by the unemployment rate in fractional form, denoted $U$. The number of the labour force in employment, denoted $N_E$, is then given by the labour force size minus the number of unemployed. That is:
\begin{align*}
N_E &= L - L \times \frac{U}{100}\\
&= L\bigg(1-\frac{U}{100} \bigg).
\end{align*}
Calculating the number of employed for each year, we get
\begin{align*}
N_E^{2006} &= 27.88 \times \bigg(1-\frac{8.7}{100}\bigg)\\
&= 25.4544...\\
N_E^{2007} &= 27.98 \times \bigg(1-\frac{9.67}{100}\bigg)\\
&= 25.2743...\\
N_E^{2008} &= 28.05 \times \bigg(1-\frac{8.65}{100}\bigg)\\
&= 25.6236...\\
N_E^{2009} &= 28.11 \times \bigg(1-\frac{8.6}{100}\bigg)\\
&= 25.6925...
\end{align*}
It is clear that the number employed is highest in the year 2009 with a value of 25.69 million. $\\$

$\textbf{Question 18} \\$ 
Using the same formula as in question 16, we know that the population in the year 2006, $P_{2006}$, is:
\begin{align*}
P_{2006} &= \frac{1.87 \times 10^{12}}{30100}\\
&= 62,126,245.847...
\end{align*}
Given that the population increased by 0.7$\%$ from 2005 to 2006, we can work out the population in 2005, as we know that:
$$P_{2006} = P_{2005} \times (1 + \frac{0.7}{100}).$$
Rearranging this in terms of $P_{2005}$ we get:
\begin{align*}
P_{2005} &= P_{2006} \times \frac{100}{100.7}\\
&= 61,694,385.151...
\end{align*}
The population in 2005 is thus $61.69 \times 10^6$. $\\$

Now we need to determine the GDP in 2005. We know the GDP in 2006 was $\$1.87 \times 10^{12}$, and the GDP rose by $2\%$ from 2005 to 2006. Working backwards again, we get that the GDP in 2005, denoted $G_{2005}$, is:
\begin{align*}
G_{2005} &= G_{2006} \times \frac{100}{102}\\
&= 1.8333 \times 10^{12}.
\end{align*}
The GDP per head of population in 2005, $G/P$, is thus:
\begin{align*}
G/P &= G_{2005} / P_{2005}\\
&= \frac{1.8333 \times 10^{12}}{61.69 \times 10^6}\\
&= 29716.3725...
\end{align*}
To the nearest hundred, we get that the solution is 29700. $\\$

$\textbf{Question 19} \\$
The total drink revenue is given by the sum of the revenue of the 5 different categories of drinks. That is:
\begin{align*}
\text{Total Revenue }&= 53+50+43+32+22\\
&= 200.
\end{align*}
The percentage of revenue made by iced tea, $R_{IT}$, is then given by the revenue of iced tea over the total revenue, then multiplied by 100 to get in percentage form. That is:
\begin{align*}
R_{IT} &= \frac{32}{200} \times 100\\
&= 16.
\end{align*}
The percentage of Kwencho's total drink revenue resulting from iced tea is thus $16 \%$. $\\$

$\textbf{Question 20} \\$
From the question, we know that the profit ratio is given by the profit as a percentage of the revenue. That is:
$$\text{Profit Ratio} = \frac{\text{Profit}}{\text{Revenue}} \times 100.$$
Applying this formula to the five drink categories we get:
\begin{align*}
\text{Cola }&: \frac{11}{53}\times 100 =20.75...\\
\text{Sparkling Water }&: \frac{12}{50}\times 100 =24\\
\text{Lemonade }&: \frac{8}{43}\times 100 =18.60...\\
\text{Iced Tea }&: \frac{-2}{32}\times 100 =-6.25\\
\text{Other }&: \frac{6}{22}\times 100 =27.27...
\end{align*}
The drinks category with the highest profit ratio is thus Other, with a percentage of 27.27$\%$. $\\$

$\textbf{Question 21} \\$
The total profit in the year 2009, denoted $P_{2009}$, is the sum of the individual profits of each category of drink, and so is given by:
$$P_{2009}= 11+12+8-2+6 = 35.$$
The total profit for the year 2010, $P_{2010}$, is then targeted to increase by 4$\%$ since 2009 and so:
$$P_{2010}= 1.04 \times 35 = 36.4.$$
The profit for the year 2011 is then targeted to increase by 5$\%$ from the profit from 2010. So we get:
$$P_{2011} = 1.05 \times 36.4 = 38.22.$$
So the targeted total profit for 2011 is £38.22 thousand, or equivalently, £38220.
\end{document}

