\documentclass{article}
\usepackage[utf8]{inputenc}
\usepackage{amsmath}

\title{Numerical Reasoning Test Solutions Test 8}
\author{Jaklyn Crilly}
\date{}

\begin{document}

\maketitle

$\textbf{Question 1} \\$
$\textbf{Note/comment: }$Question asks `in 2009', but there is no year labelled anywhere on the data, so we don't actually know that the data is for the year 2009. $\textbf{Note/comment end.}$ $\\$

The turnover for the Redbrook site in 2009 was £589000, and there was a total of 50 staff members there. $\\$

The turnover per employee at the Redbrook site, denoted $T/E_R$, is given by the total turnover divided by the total number of staff, and so:
\begin{align*}
T/E_R &= \frac{589000}{50}\\
&= 11780.
\end{align*}
To two decimal places, the answer is £11780.00 per employee. $\\$

$\textbf{Question 2} \\$
The turnover per employee for the whole company, denoted $T/E$, is the sum of the turnover for each site, divided by the sum of the number of staff at each site.$\\$

The total turnover for the company is:
$$378000 + 589000 + 433000 = 1370000.$$

The total number of staff at the company is:
$$25 + 50 + 25 = 100.$$

The turnover per employee for the whole company is then:
\begin{align*}
T/E &= \frac{1400000}{100}\\
&=14000.
\end{align*}

The answer is £14000 per employee. $\\$

$\textbf{Question 3} \\$
We have not been given any data in relation to the total number of staff in the company in 2010, and so we cannot work out the total turnover. As a result, we cannot work out by how much the turnover increased from 2009 to 2010.$\\$

$\textbf{Question 4} \\$
The percentage of people who responded `Probably not' was 4$\%$. Given there was a total number of 522000 responses, the number of `Probably not' responses, denoted $N_{PN}$, is:
\begin{align*}
N_{PN} &= 522000 \times \frac{4}{100}\\
&= 20880.
\end{align*}
The answer is 20880 people.  $\\$

$\textbf{Question 5} \\$
The percentage of people who responded `Definitely not' was 2$\%$. Given there was a total number of 522000 responses, the number of `Definitely not' responses, denoted $N_{DN}$, is:
\begin{align*}
N_{DN} &= 522000 \times \frac{2}{100}\\
&= 10440.
\end{align*}
Of the 10440 people who responded `Definitely not', 5$\%$ moved to different doctors, meaning 95$\%$ remained with the same doctor. The number of these people that stayed, denoted $N_{DN}^{\text{stay}}$, is then:
\begin{align*}
N_{DN}^{\text{stay}} &= 10440 \times \frac{95}{100}\\
&= 9918.
\end{align*}
So 9918 of the people who responded `Definitely not' remained with their original doctor. $\\$

$\textbf{Question 6} \\$
62$\%$ of the people who took the survey responded `Yes definitely' to this particular question. This means that $100\%-62\% = 38\%$ of the people responded with an answer different to `Yes definitely'. Of this 38$\%$ of people, one quarter of them agreed to do a second test. The overall percentage of people who did not respond `Yes definitely', and agreed to take the second test is thus:
$$38\% \times 0.25 = 9.5\%.$$
Of this $9.5\%$, 20 $\%$ did not return there forms, which means that 80$\%$ of them did return their forms. So the total percentage of people who did not respond `Yes definitely', agreed to take the second test, and $\textbf{did}$ return their forms is:
$$9.5\% \times \frac{80}{100} = 7.6 \%.$$
The answer is 7.6$\%$. $\\$

$\textbf{Question 7} \\$
From the graph we see that in July, £1 is equal to $\$$1.48, so we get:
$$£1 = \$1.48.$$
Multiplying both sides of this equation by 10:
$$£10 = \$14.8.$$
Therefore in July, £10 is worth $\$$14.80. $\\$

$\textbf{Question 8} \\$
One way we could solve this question is to first calculate the difference between the US dollar value of £10 and the US dollar value of 10 SFr for each month, and then take the average of these values. A quicker and simpler way to solve this problem however, is to calculate the average US dollar value of £10 and the average US dollar value of 10 SFr over the given months, and then calculate their difference. $\\$

Approaching this question using the latter method, letting $A_{\text{currency}}$ denote the average US dollar value for 1 unit of currency from the months of January to May, we get:\\
\begin{align*}
A_{\text{£}} &= \frac{1.54+1.54+1.57+1.54+1.48}{5}\\
&= 1.534,\\
A_{\text{SFr}} &= \frac{0.94+0.95+0.98+0.98+0.98}{5}\\
&= 0.966.
\end{align*}
Multiplying by 10, we see that the average US dollar value for £10 is $\$$15.34, and the average US dollar value for 10 SFr is $\$$9.66.

The average difference is thus:
$$\$15.34-\$9.66 = \$5.68.$$

$\textbf{Question 9} \\$
For the month of August, we see that €1 is worth $\$$1.23. Multiplying both values by 100 we get:
$$€100 = \$123.$$
Now from the graph we also see that £1 is worth $\$$1.45. Multiplying both values by 123 and then dividing by 1.45 we get:
\begin{align*}
\$123 &= £1 \times \frac{123}{1.45}\\
&= £84.8275...
\end{align*}
So we see that €100 is worth $\$$123, which in turn is worth (to two decimal places) £84.83. $\\$

$\textbf{Question 10} \\$
In December, we have that £1 is worth $\$$1.48. If the pound rose by 3$\%$ by the following January, we get that in January:
\begin{align*}
£1 &= \$1.48 \times 1.03\\
&= \$1.5244.
\end{align*}
So after this 3$\%$ rise, £1 is worth $\$$1.5244. Multiplying both sides of this equation by 200 and then dividing by 1.5244, we get:
\begin{align*}
\$200 &= £1 \times \frac{200}{1.5244}\\
&= £131.1991...
\end{align*}
So, to two decimal places, $\$$200 is now worth £131.20. $\\$

$\textbf{Question 11} \\$
The units of currency are shown as £ million in the table, and so given the tables value for `profit before interest and taxation' in 2007/8 is 921, this means that there was £921 million of profit before interest and taxation.  $\\$

$\textbf{Question 12} \\$
The turnover in 2007/8 was £4970 million, whilst the turnover in 2006/7 was £5268 million. Therefore, the turnover decrease, $T_D$, was:
\begin{align*}
T_D &= 5268 \times 10^6 - 4970 \times 10^6\\
&= 298 \times 10^6.
\end{align*}
So the turnover decreased by $£298\times 10^6$.$\\$

The percentage decrease in turnover, denoted $P_D$, is then given by expressing the turnover decrease as a percentage over the initial turnover (that is, the turnover for 2006/7). So we get:
\begin{align*}
P_D &= \frac{298}{5268} \times 100\\
&= 5.6567...
\end{align*}
The answer to to one decimal place is 5.7$\%$. $\\$

$\textbf{Question 13} \\$
We know that the Gross Profit Margin is given by the Gross Profit as a percentage of the turnover. Letting $\text{GPM}_{year}$ denote the Gross Profit Margin for the specified year, we get:
\begin{align*}
GPM_{2006/7} &= \frac{893}{5268} \times 100\\
&= 16.95...,\\
GPM_{2007/8} &= \frac{950}{4970} \times 100\\
&= 19.11....
\end{align*}
The difference in the Gross Profit Margin for 2006/7 and 2007/8 is:
$$19.11...-16.95... =2.16... $$
So to one significant figure, we get that the answer is 2$\%$. $\\$

$\textbf{Question 14} \\$
$\textbf{Note/comment: }$Graph should have the units labelled on the vertical axis as ($\%$). Also, wouldn't hurt to label the x-axis `Years' $\textbf{Note/comment end}$ $\\$

The index for employment in Transport in the year 2000 is 100$\%$, and in the year 2006 is 103$\%$.

The change in employment in Transport index from the years 2000 to 2006 is given by their difference (that is, the value for the year 2006 minus that of the year 2000), and so it is $103\%-100\% = 3\%$. $\\$

The percentage increase for employment in Transport, denoted $P_T$, is then given by the change in the index for employment in Transport expressed as a percentage of the initial employment in Transport index (the index for the year 2000). So we get:
\begin{align*}
P_T &= \frac{3}{100} \times 100\\
&= 3.
\end{align*}
Therefore, the answer is 3$\%$. $\\$

$\textbf{Question 15} \\$
Applying the exact same steps as we did for question 14, we know that the Business Services index in the year 1997 was 86$\%$, and in the year 2008 was 121$\%$. The increase in the Business Services index between 1997 and 2008 is then $121\% - 86 \% = 35\%$. $\\$

So the percentage increase for employment in Business Services, $P_B$, is:
\begin{align*}
P_B &= \frac{35}{86}\times 100\\
&= 40.6976...
\end{align*}
To the nearest percent, we get that the answer is 41$\%$. $\\$

$\textbf{Question 16} \\$
The index only tells us the percentage growth based on the year 2000. We have not been given any information in regards to the number of people employed in any of the services for any year. Therefore we cannot say which service had the greatest number of people employed. $\\$

$\textbf{Question 17} \\$
We need to rebase all the Industry employment indices using 2005 as the new base. This means that we have to divide all of the Industry employment indices by the value of the Industry employment index in the year 2005 (which is 94$\%$), and then multiply them all by 100 to get the values back into percentage form. (Note that this process results in the year 2005 having an index of 100$\%$, as required). $\\$

Given the Industry employment index for the year 2000 was 100$\%$, after we have rebased, the new Industry employment index for the year 2000 will be:
$$100 \times \frac{100}{94} = 106.38.$$
The answer to the nearest whole number is thus 106$\%$. $\\$

$\textbf{Question 18} \\$
We simply need to sum together the exports (which is indicated by the purple colour) for Poland and Bulgaria. Doing so, we get:
$$54+23=77.$$
Given the units are in € millions, we get that the combined value of Grossland's exports to Poland and Bulgaria was €77 million. $\\$

$\textbf{Question 19} \\$
We need to calculate the value of imports that came from the Czech Republic as a percentage of the value of the total imports. The value of imports (in units of € millions) that came from the Czech Republic is 24, whilst the total value of imports, which is the sum of all the individual countries imports, is:
$$17+24+11+18+50 = 120.$$
So the percentage of Grossland's total imports coming from the Czech Republic, denoted $P_{CR}$, is:
\begin{align*}
P_{CR} &= \frac{24}{120} \times 100\\
&= 20.
\end{align*}
The answer is 20$\%$. $\\$

$\textbf{Question 20} \\$
The value of exports to Romania in 2007 was €56 million. If this value fell by 0.5$\%$ between 2007 and 2008, we get that the value of exports to Romania in 2008, denoted $E_R$, is:
\begin{align*}
E_R &= 56 - 56\times \frac{0.5}{100}\\
&= 55.72.
\end{align*}
The answer is thus €55.72 million. $\\$

$\textbf{Question 21} \\$
Grossland's total exports for the year 2007, denoted $E_G^{2007}$, is given by the sum of the value of the individual exports to each country, so we get:
\begin{align*}
E_G^{2007} &= 54+47+23+56+20\\
&= 200.
\end{align*}
Given that in 2007, the value of Grossland's exports was 25$\%$ higher than that of Nettland's exports, which we will denote by $E_N^{2007}$, we know that:
$$E_G^{2007} = E_N^{2007} \times 1.25.$$
Rearranging this equation in terms of $E_N^{2007}$, we get:
\begin{align*}
E_N^{2007} &= 200 \times \frac{1}{1.25}\\
&= 160.
\end{align*}
The total value of Nettland's exports in the year 2007 was €160 million.$\\$

Now Nettland's exports rose by 5$\%$ from 2007 to 2008. Given in 2007 their exports totaled €160 million, we get that in 2008, the value of Nettland's exports (in units € millions) is:
\begin{align*}
E_N^{2008} &= 160 + 160 \times \frac{5}{100}\\
&= 168.
\end{align*}
So the answer is €168 million.

\end{document}

