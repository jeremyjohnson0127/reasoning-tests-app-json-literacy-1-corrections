\documentclass{article}
\usepackage[utf8]{inputenc}
\usepackage{amsmath}

\title{Numerical Reasoning Test Solutions Test 9}
\author{Jaklyn Crilly}
\date{}

\begin{document}

\maketitle

$\textbf{Question 1} \\$
In June 2010, 1$\%$ of the 8500 candidates were unclassified. Therefore, we get that the number of unclassified candidates, denoted $N_U$, is:
\begin{align*}
N_U &= 8500 \times \frac{1}{100}\\
&=85.
\end{align*}
The answer is 85 candidates. $\\$

$\textbf{Question 2} \\$
In June 2007, 44$\%$ of the 6000 candidates achieved an A-grade, whilst in June 2010, 23$\%$ of the 8500 candidates achieved an A-grade. Letting $N_A^{\text{year}}$ denote the number of candidates that achieved an A-grade in June of the specified year, we get:
\begin{align*}
N_A^{2007} &= 6000 \times \frac{44}{100}\\
&= 2640,\\
N_A^{2010} &= 8500 \times \frac{23}{100}\\
&= 1955.
\end{align*}
The change in these numbers from 2007 to 2010 is thus:
\begin{align*}
N_A^{2010} - N_A^{2007} &= 1955 - 2640\\
&= -685.
\end{align*}
So the answer is -685 candidates. $\\$

$\textbf{Question 3} \\$
$\textbf{Comment: }$ I think it should say `in' instead of `on' at the end of the first line. $\textbf{Comment end.}$ $\\$

In June 2010, 5$\%$ of the students that scored a grade of `D' took up places in undergraduate Business Studies courses at university. $\\$

Now of this 5$\%$, we know that 20$\%$ failed to graduate which means that 80$\%$ succeeded in graduating. Out of the number of students who scored a D-grade, the percentage that took up places in Business Studies courses and did graduate is thus:
$$5\% \times \frac{80}{100} = 4\%.$$
So the answer is 4$\%$. $\\$

$\textbf{Question 4} \\$
As given in the question, we know that payments change in proportion to the Retail Price Index (RPI). Given we want to determine how much a payment of £150 in 1998 would have been in 2009, we first need to rebase the RPI data so that the base index of 100$\%$ occurs in 1998. This is achieved by dividing all RPI values by the 1998 Retail Price Index (which is 162.9$\%$), and then multiplying by 100 to get the data back into percentage form. Doing this for the year 2009, we get that the new RPI for this year is:
\begin{align*}
\text{RPI}_{2009} &= \frac{213.7}{162.9} \times 100\\
&= 131.1847...
\end{align*}
So the payments of £150 in the year 1998, will now cost 131.18$\%$ of the original £150 in the year 2009. So the new payments, denoted $P_{2009}$, will be:
\begin{align*}
P_{2009} &= 150 \times \frac{131.18}{100}\\
&= 196.7771...
\end{align*}
So to two decimal places, the payments cost £196.78. $\\$

$\textbf{Question 5} \\$
As given in the question, we know that payments change in proportion to the Consumer Price Index (CPI). Using the same process as we did for the previous question, we need to rebase all the CPI values so that the base index of 100$\%$ occurs in the year 2006. This is achieved by dividing all CPI values by 102.3 (the CPI for the year 2006), and then multiplying all values by 100 in order to get the data back into percentage form. Doing this for the year 2000, we get that the new CPI for this year is:
\begin{align*}
\text{CPI}_{2000} &= \frac{93.1}{102.3}\times 100\\
&= 91.0068...
\end{align*}
Therefore, the rebased CPI for the year 2000 is 91.01$\%$, and so we get that the value of the payments in 2000 is 91.01$\%$ of the value of the payments in 2006. The payments in 2000, denoted $P_{2000}$, is thus:
\begin{align*}
P_{2000} &= 300 \times \frac{91.0068}{100}\\
&= 273.0205...
\end{align*}
So to two decimal places, the payments cost £273.02 in the year 2000. $\\$

$\textbf{Question 6} \\$
Using the same formulae as we did in the previous two questions, we know that the RPI in 2000 was 170.3$\%$ and the CPI in 2000 was 93.1$\%$, so rebasing the two sets of data to the year 2000, we get that the rebased RPI and CPI for the year 2009, denoted $\text{RPI}_{2009}$ and $\text{CPI}_{2009}$ respectively, is:
\begin{align*}
\text{RPI}_{2009} &= \frac{213.7}{170.3}\times 100\\
&=125.4844...\\
\text{CPI}_{2009} &= \frac{110.8}{93.1}\times 100\\
&= 119.0118...
\end{align*}
Therefore, by adjusting the benefit amount in line with the RPI and CPI respectively, a benefit payment of £200 in the year 2000 will, in the year 2009, cost:
\begin{align*}
P_{2009}^{\text{RPI}} &= 200 \times \frac{125.4844}{100}\\
&= 250.48444...\\
P_{2009}^{\text{CPI}} &= 200 \times \frac{119.0118}{100}\\
&= 238.0236...,
\end{align*}
Deducting $P_{2009}^{\text{RPI}}$ from $P_{2009}^{\text{CPI}}$, we get:
\begin{align*}
P_{2009}^{\text{RPI}} - P_{2009}^{\text{CPI}} &=250.48444... -  238.0236...\\
&= 12.9452...,
\end{align*}
and so we see that if the benefit amount was adjusted annually in line with the CPI, in 2009 the payment would be (to the nearest £0.01) £12.95 less than they would have been if it was adjusted in line with the RPI. $\\$

$\textbf{Question 7} \\$
From the graph we see that £1 is worth 5.01 PLN on the 27/06/2010. Multiplying both values by 350, we get:
\begin{align*}
£350 &= 5.01 \times 350 \text{ PLN}\\
&= 1753.5 \text{ PLN}.
\end{align*}
So we see that £350 is worth 1735.50 PLN. $\\$

$\textbf{Question 8} \\$
$\textbf{Comment: }$You have written Swiss Franc's unit as CGF instead of CHF in the question (written correctly in table). Also you have a `1' at the end of the question, right before the question mark. I don't think that is meant to be there. $\textbf{Comment end.}$ $\\$

We know that £1 is worth 1.79 CHF, so we get that:
$$1.79 \text{ CHF} = £1.$$
Multiplying both sides of the equation by 2000 and then dividing by 1.79 we get:
\begin{align*}
2000 \text{ CHF} &= £1 \times \frac{2000}{1.79}\\
&= £1117.3184...
\end{align*}
Therefore, to two decimal places, we see that 2000 CHF is worth £1117.32. $\\$

$\textbf{Question 9} \\$
We have that on the 27/06/2010, £1 is worth $\$$1.49. Multiplying both of these values by 200 and dividing by 1.49 we get:
\begin{align*}
\$200 &= £1 \times \frac{200}{1.49}\\
&= £134.2281...
\end{align*}
We also know that on the 27/06/2010, £1 is worth 9.50 NOK. Multiplying both of these values by $\frac{200}{1.49} = 134.2281...$, we get:
\begin{align*}
£134.2281... &= 9.50 \times \frac{200}{1.49} \text{ NOK}\\
&= 1275.1677... \text{ NOK}
\end{align*}
So to two decimal places, we see that $\$$200 is worth £134.23, which in turn is worth 1275.17 NOK. $\\$

$\textbf{Question 10} \\$
We need to work out what €1 is worth in terms of the US dollar for the dates 27/06/2009 and 27/06/2010. We then must work out the change in these values over this year (given by the 2010 term minus the 2009) as a percentage of the initial value (that is, the worth of €1 in term of the US dollars in 27/06/2009). $\\$

We know that on the 27/06/2009, $€1.17 = £1 = \$1.65$. Dividing this equation by 1.17, we get:
\begin{align*}
€1 &= \$1.65 \times \frac{1}{1.17}\\
&= \$1.4102...
\end{align*}
Repeating this for the date 27/06/2010, we know that $€1.21 = £1 = \$1.49$. Dividing this equation by 1.21, we get:
\begin{align*}
€1 &= \$1.49 \times \frac{1}{1.21}\\
&= \$1.2314...
\end{align*}
The percentage change in the value of €1 measured against the US dollar between 27/06/2009 and 27/06/2010, denoted $P_C$, is:
\begin{align*}
P_C &= \frac{1.2314...-1.4102...}{1.4102...} \times 100\\
&= -12.6821...\%.
\end{align*}
The negative sign indicates a price fall, and so we get that to the nearest percent, the answer is a 13$\%$ fall. $\\$

$\textbf{Question 11} \\$
There are 945 business founders in Northland that are less than 30 years old, whilst the total number of business founders in Northland is:
$$2052+1503+945 = 4500.$$
The percentage of business founders in Northland that are less than 30 years old, denoted $P_{30}^N$, is then:
\begin{align*}
P_{30}^N &= \frac{945}{4500} \times 100\\
&= 21.
\end{align*}
The answer is therefore 21$\%$. $\\$

$\textbf{Question 12} \\$
There are $1071+1176 = 2247$ business founders in Southland that are older than 30, whilst the total number of business founders in Southland is:
$$753+1071+1176 = 3000.$$
The percentage of business founders in Southland that are less than 30 years old, denoted $P_{30}^N$, is then:
\begin{align*}
P_{30}^N &= \frac{2247}{3000} \times 100\\
&= 74.9 \%. 
\end{align*}
The answer is therefore 74.9$\%$. $\\$

$\textbf{Question 13} \\$
Letting $N_{\text{Region}}$ denote the number of founders for the specified region (taking S=Southland, N=Northland and M=Midway), we get:
\begin{align*}
N_S &= 753 + 1071 + 1176\\
&= 3000,\\
N_M &= 1056 + 1716 + 3228\\
&=6000,\\
N_N &= 945 + 1503 + 2052\\
&= 4500.
\end{align*}
The total number of founders across all regions, denoted $N$, is then given by the sum of the total number of founders over the three regions, and so:
\begin{align*}
N &= N_S + N_M + N_N\\
&= 3000 + 6000 + 4500\\
&= 13500.
\end{align*}
The youngest 10$\%$ of the total 13500 business founders would receive the award. Thus, the total number of business founders that would receive the award, denoted $N^{\text{award}}$, is:
\begin{align*}
N^{\text{award}} &= 13500 \times \frac{10}{100}\\
&= 1350.
\end{align*}
So the award of £1500 would be awarded to 1350 business founders, which means that a total amount of $£1500 \times 1350 = £2,025,000 =£2.025 \text{ million}$ will be needed for such a scheme. $\\$

$\textbf{Question 14} \\$
The GDP per head of population is given by the GDP divided by the total population. For the case of France, we see that the GDP is $\$$2,247,000 million, and the total population is 61 million. The GDP per head of population, denoted $\text{GDP}/P$, is thus:
\begin{align*}
\text{GDP}/P &= \frac{2247000}{61}\\
&= 36836.0655...
\end{align*}
(Note that we could have included the factors of $10^6$ (resulting due to the units being in millions) for both the GDP and the population, however, given we are dividing these quantities together these factors would cancel each other, and so they can be ignored).$\\$

The answer to the nearest whole number is $\$$36836 per person. $\\$

$\textbf{Question 15} \\$
Letting $P_{W}$ denote the working age population, and $r_E$ the employment rate (which is expressed as a fraction, that is, the employment rate in percentage form divided by 100), the total number of the population employed, denoted $P_E$, is given by:
$$P_E = P_{W} \times r_E.$$
The GDP per head of employed population, $\text{GDP}/P_E$, is then given by the GDP divided by the total number of the population that is employed. That is:
\begin{align*}
\text{GDP}/P_E &= \frac{\text{GDP}}{P_E}\\
&= \frac{\text{GDP}}{P_{W} \times r_E},
\end{align*}
where here the GDP and the working age population are in units of millions.$\\$

Calculating this number for the 4 countries, we get:
\begin{align*}
(\text{GDP}/P_E)_{\text{Canada}} &= \frac{990 \times 10^3}{22 \times 0.655}\\
&=68702.2900...\\
(\text{GDP}/P_E)_{\text{France}} &= \frac{2247 \times 10^3}{40 \times 0.64}\\
&=87773.4375,\\
(\text{GDP}/P_E)_{\text{Germany}} &= \frac{2943 \times 10^3}{50 \times 0.669}\\
&=87982.0627...\\
(\text{GDP}/P_E)_{\text{Japan}} &= \frac{4890 \times 10^3}{77 \times 0.703}\\
&=90336.4061...
\end{align*}
From this data, it is clear that Japan has the highest GDP per head of employed population, with a value of $\$$90336.41 per person. $\\$

$\textbf{Question 16} \\$
The mobile phone plan is a 24 month contract. The base cost of the plan for this period (that is, the cost of the plan with no additional costs resulting due to extra call minutes or texts), denoted $C^{\text{base}}$, is thus the cost of the phone plus 24 times the value of the monthly tariff. So, for plan C we get:
\begin{align*}
C^{\text{base}}_C &= 119 + 24 \times 35\\
&= 959.
\end{align*}

So the base cost of the phone over the 24 month contract is £959. Given plan C offers 600 free call minutes per month, and an unlimited number of texts, if the customer used 500 call minutes and 70 texts each month, they will not have to pay any extra fee (as this usage is covered in their free call minutes and texts). Therefore, the total cost of plan C over the full contract for this customer would be £959. $\\$

$\textbf{Question 17} \\$
As in the previous question, we will first determine the base cost of plan A over the 24 month contract, denoted $C_A^{\text{base}}$. This is given by:
\begin{align*}
 C_A^{\text{base}} &= 219 + 24 \times 25\\
&= 819.
\end{align*}
So the cost of plan A over the 24 month contract (without any extra fees) is £819. $\\$

If a customer on plan A used 500 call minutes and 300 texts each month, given plan A offers 75 call minutes and 250 texts for free each month, the customer will have to pay an extra fee for $500-75=425$ call minutes and $300-250=50$ texts monthly. Given the extra fee is $30\text{p}=£0.3$ per minute and $24\text{p}=£0.24$ per text, each month the customer will have to pay (in addition to the base cost) an extra:
$$0.3 \times £425 + 0.24 \times £50 = £139.5.$$
Given the contract is 24 months long, the total additional fees the customer will have to pay is:
$$£139.5 \times 24 = £3348.$$

The total price of plan A over the full length of the contract for this customer, is then given by the sum of the base price and the additional fees, and is thus:
$$£819 + £3348 = £4167.$$ 

$\textbf{Question 18} \\$
It is not possible to calculate the cost of extra call minutes if only the average usage is given. For an example detailing why, consider the following two customers on plan F. $\\$

 The first customer used exactly 3010 call minutes each month. So each month they had made 10 minutes of extra calls, and so the additional fees that they would have had to pay is $£10 \times £0.3=£3$ per month, and thus,
 an additional $£3 \times 24 = £72$ in total for the full contract period. $\\$

Now consider a second customer who made no calls in the first 12 months and then 6020 minutes worth of calls in the remaining 12 months (note that this still averages 3010 calls per month). Then there will be no additional fees for the first 12 months, but for the following 12 months, they will have made $6020-3000=3020$ extra call minutes monthly, and so each month they will have to pay an extra $3020 \times £0.3 = £906$. Given this occurs for 12 months, this means the total additional fees they have to pay is $£906 \times 12 = £10872$.$\\$

It is clear that in the second case they have to pay much more, however in both cases the average call minutes per month is the same. $\\$

$\textbf{Question 19} \\$
From the graph, it is clear that the revenue for Iced Tea (coloured green) is always well below the revenue for Fruit Drinks (coloured purple) over the whole year. Therefore, throughout 2007 the revenue for Iced Tea never exceeds the revenue for Fruit Drinks, and so the answer is 0.$\\$

$\textbf{Question 20} \\$
We will first work out the total revenue (which is the sum of the revenues of the four drink categories) made in November, which we denote $R_N$, and December, $R_D$. We will then calculate their percentage change, which is the change in total revenue from November to December, expressed as a percentage over the initial month's total revenue (i.e. November's).$\\$

The total revenue (in units of £ millions) made in November, and December, is:
\begin{align*}
R_N &= 3.1 + 5.0 + 8.4 + 11.0\\
&= 27.5,\\
R_D &= 2.5 + 4.8 + 8.7 + 11.6\\
&= 27.6.
\end{align*}
The percentage change in the total revenue, denoted $P_C$, is thus:
\begin{align*}
P_C &= \frac{27.6-27.5}{27.5} \times 100\\
&= 0.3636...
\end{align*}
So the percentage change, to two decimal places, is 0.36$\%$. Given this is a positive value, we get that the revenue had a 0.36$\%$ rise from November to December. $\\$

$\textbf{Question 21} \\$
The revenue for Iced Tea in January 2007 was £2.3 million. Given the revenue from December 2006 to January 2007 rose by 3$\%$, letting $R_{\text{Dec06}}$ denote the revenue made in December 2006 (in units of £ millions), we know that:
$$2.3 = R_{\text{Dec06}} \times \frac{100 + 3}{100}.$$
Rearranging this so that $R_{\text{Dec06}}$ is the subject, we get:
\begin{align*}
R_{\text{Dec06}} &= 2.3 \times \frac{100}{103}\\
&= 2.2330...
\end{align*}
Given the revenue increased by 1$\%$ from November 2006 to December 2006, applying the same process again, we get:
\begin{align*}
R_{\text{Nov06}} &= R_{\text{Dec06}} \times \frac{100}{101}\\
&= 2.2109...
\end{align*}
So we see that the revenue in November 2006, to four significant figures, is:
\begin{align*}
£2.211\text{ million} &= £2.211 \times 10^6 \\
&=£2.211 \times 10^3 \text{ thousand}.
\end{align*}
The answer is therefore $£2.211 \times 10^3 \text{ thousand} =£2211\text{ thousand}$. 
\end{document}

